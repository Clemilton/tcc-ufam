\pagenumbering{arabic}

\chapter{Introdução} %Contextualização, motivação e justificativa.

Séries estão presentes em diversos fenômenos ao nosso redor. Tanto em atividades humanas como processos naturais, as séries temporais estão presentes em domínios como: clima, economia, medicina, transporte. Avanços rápidos em tecnologias, desde sensores remotos a \textit{wearables}, estão gerando um rápido crescimento no tamanho e na complexidade dos fluxos de dados de séries temporais. Assim, a importância e o impacto das técnicas de análise e modelagem de séries temporais continuam a crescer.

Dentre as diferentes análises, a Classificação de Séries Temporais (\textit{TSC}) tem sido considerada um dos problemas mais desafiadores na mineração de dados. Uma característica que não pode ser ignorada é a correlação entre as amostras. Com o aumento de dados temporais, centenas de métodos para \textit{TSC} foram propostos desde 2015 \cite{Bagnall2017}. A abordagem mais popular e tradicional é o k-vizinhos mais próximos(KNN) combinada com a distância \textit{Dynamic Time Warping} (DTW). A ideia principal consiste em fazer o cálculo da distância 

O objetivo deste trabalho é avaliar diferentes arquiteturas de aprendizagem profunda para classificação de séries temporais com base no repositório da UCR.

Dentro dessa realidade, o presente projeto aborda o desenvolvimento de um sistema de detecção de f0 para áudios musicais monofônicos, através de processamento digital de sinais, com o intuito de mapear os áudios, fornecendo as frequências fundamentais soadas em cada instante de tempo e identificando os fenômenos aos quais um sistema nesse contexto é submetido. A detecção de f0 será realizada por meio da Transformada de Fourier de Curto Tempo(STFT), considerando f0 como a frequência de maior amplitude dentro de cada janela. Foi construída uma base de dados com áudios de 4 tipos de instrumentos musicais para realização de experimentos, visando validar as detecções realizadas pelo sistema. Este trabalho propõe-se ainda a avaliar a metodologia adotada por meio desses experimentos realizados. 
\section{Organização do Trabalho}



%O texto de introdução deve conter três tipos de informações: apresentação do problema, estado da arte e justificativa do projeto.
%A apresentação ou formulação do problema deve deixar, de forma bem clara, qual será o objeto de estudo do projeto. As razões para a escolha do tema deverão ser justificadas. Desta forma devem ser indicadas: a importância do estudo, quais as possíveis repercussões, quais hipóteses a serem verificadas, entre outras. 
%O estado da arte serve para embasar tanto a formulação do problema como sua justificativa. É preciso situar historicamente a evolução do tema, quais as abordagens já investigadas, qual o estágio atual do conhecimento sobre o assunto ou quais as tendências que se apresentam. Indique as palavras-chave que foram utilizadas para a pesquisa bibliográfica.
%A justificativa do projeto deve indicar por que o projeto deve ser feito, ou seja, por que o problema tratado é relevante. Descreva os fatores de motivação que o(s) levaram a abordar e trabalhar no assunto.
%O final da introdução deve incluir uma descrição de como o documento está estruturado (um parágrafo para indicar o conteúdo de cada seção do Plano do Projeto).

\section{Objetivo}

% Fim Capítulo