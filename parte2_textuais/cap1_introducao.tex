\pagenumbering{arabic}

\chapter{Introdução} %Contextualização, motivação e justificativa.

% [Contemporâneo - Como as tecnologias ajudam na música (de forma geral)]

\section{Organização do Trabalho}



%O texto de introdução deve conter três tipos de informações: apresentação do problema, estado da arte e justificativa do projeto.
%A apresentação ou formulação do problema deve deixar, de forma bem clara, qual será o objeto de estudo do projeto. As razões para a escolha do tema deverão ser justificadas. Desta forma devem ser indicadas: a importância do estudo, quais as possíveis repercussões, quais hipóteses a serem verificadas, entre outras. 
%O estado da arte serve para embasar tanto a formulação do problema como sua justificativa. É preciso situar historicamente a evolução do tema, quais as abordagens já investigadas, qual o estágio atual do conhecimento sobre o assunto ou quais as tendências que se apresentam. Indique as palavras-chave que foram utilizadas para a pesquisa bibliográfica.
%A justificativa do projeto deve indicar por que o projeto deve ser feito, ou seja, por que o problema tratado é relevante. Descreva os fatores de motivação que o(s) levaram a abordar e trabalhar no assunto.
%O final da introdução deve incluir uma descrição de como o documento está estruturado (um parágrafo para indicar o conteúdo de cada seção do Plano do Projeto).

\section{Objetivo}

% Fim Capítulo