\chapter{Conclusão} \label{cap6}

%% O que foi feito neste projeto?
O presente projeto implementou um sistema de detecção de frequências fundamentais para áudios musicais, baseado na análise por janelas de curto tempo. A fim de avaliar o sistema desenvolvido, foi construído uma base de dados contendo áudios gravados com 4 instrumentos musicais diferentes e seus respectivos registros temporais de nota, e foi desenvolvido um ambiente de experimentação com interface gráfica para a seleção dos áudios da base, e geração dos gráficos para a avaliação de desempenho da detecção. Por fim, foram executados os experimentos e calculado a métrica de percentual de sucesso nas detecções, avaliou-se os resultados obtidos identificando também os eventuais problemas encontrados pelo sistema no momento da detecção. Discutiu-se também a metodologia adotada, buscando evidenciar os aspectos positivos e negativos da mesma, bem como as melhorias que poderiam ser incorporadas a fim de tornar o sistema mais robusto.


%% Contribuição deste projeto?
Os conceitos abordados neste trabalho, relacionados ao processamento e análise de sinais digitais, colaboraram para uma melhor compreensão do tema, e permitindo um melhor desenvolvimento do projeto. O desenvolvimento deste proporcionou um aprofundamento prático em relação à detecção de $f_0$ no ambiente musical e pode-se observar as diferentes características do som de cada instrumento musical quando na análise pelo domínio da frequência. Também verificou-se que métodos de detecção de $f_0$ baseados na obtenção da frequência de maior amplitude estão vulneráveis aos efeitos de ressonância, sendo necessário um tratamento específico para este problema, entretanto, o método apresentou bons resultados quando em sons não muito graves. Outra importante contribuição deste projeto é a interface gráfica de experimentação, que mostrou-se prática e eficaz para a avaliação do método de detecção adotado, e pode ser aproveitada em outros trabalhos neste tema.

%% Dificuldades encontradas?


%% Quais os próximos passos para o desenvolvimento deste projeto?
Espera-se que este trabalho seja um incentivo para novos projetos em Processamento Digital de Sinais voltados para a área musical. Muitos outros métodos de detecção de frequência fundamental tem sido apresentados e desenvolvidos, sendo necessário o trabalho de comparar esses diferentes métodos para obter-se o mais apropriado para a detecção desejada. Outro esforço também pode ser empreendido na intenção de reforçar o método adotado neste projeto, para que se alcance bons resultados mesmo em análise de sons graves.

% Fim Capítulo
