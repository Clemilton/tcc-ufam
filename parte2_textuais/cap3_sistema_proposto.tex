\chapter{Arquiteturas Propostas} \label{cap3}

Neste capítulo são expostas as arquiteturas de redes profundas utilizadas no trabalho.


\section{Visão geral}


\section{Perceptron Multicamadas}
A primeira rede implementada foi o Perceptron Multicamadas. A rede contém: uma camada de entrada, três camadas ocultas e uma camada de saída. A camada de entrada tem o mesmo tamanho da série temporal. As camadas ocultas são compostas por 500 neurônios com a ReLu como função de ativação. A última camada é uma Softmax e contém o número de neurônios igual a quantidade de classes do dataset.

Entre as camadas é utilizada a técnica de \textit{Dropout} com uma taxa de 0.1, 0.2, 0.2 e 0.3, depois da primeira, segunda, terceira e quarta camada respectivamente. Essas taxas determinam uma porcentagem de neurônios que são desativados durante o treinamento em cada época. \textit{Dropout} é uma forma de regularização que ajuda na prevenção do overfitting \cite{dropout2014}. 


\section{Fully Convolutional Networks}

http://principlesofdeeplearning.com/index.php/a-tutorial-on-global-average-pooling/


