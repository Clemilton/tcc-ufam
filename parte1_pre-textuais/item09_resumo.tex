\chapter*{Resumo}
\thispagestyle{empty}

Existem diferentes caminhos para o reconhecimento automático de uma notas musical, e um deles é por meio da frequência fundamental ($f_0$). Muitas soluções e métodos de detecção automática de $f_0$ já foram desenvolvidas e apresentadas, obtendo resultados positivos. Todavia, é difícil que uma única solução de detecção seja de fato eficaz em ambientes muito diferentes daqueles para o qual a solução foi proposta. Diante disso, o presente projeto aborda o desenvolvimento de um sistema de detecção de $f_0$ para áudios musicais monofônicos, através de processamento digital de sinais, com o intuito de mapear os áudios, fornecendo as frequências fundamentais soadas em cada instante de tempo. Este trabalho também avaliou o sistema desenvolvido, por meio de experimentação a partir de uma base de dados construída para este fim. Percebeu-se que o sistema proposto apresenta boas respostas, sendo necessário melhorias em relação ao tratamento do efeito de ressonância.

\vspace{50pt}

\paragraph{Palavras-chave:} Detecção automática de frequência fundamental, Processamento digital de áudio, Transcrição musical.
